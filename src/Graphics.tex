\part{AutoCAD图形管理}

在AutoCAD中,图形文件的管理是通过图形文件管理器来操作的,包括新建图形文件、打开图形文件及保存图形文件的操作,其中创建、打开和关闭图形文件是绘制图形的基础。另外,AutoCAD还提供了图形修复、安全口令和数字签名等提高文件管理安全性。

\chapter{建立新图形文件}

在默认状态下,AutoCAD执行NEW命令打开“选择样板”对话框。在“选择样板”对话框的文件列表框中,可以选择其中的某一个样板文件作为样板来创建新图形,“选择样板”对话框的文件列表框中提供了三种类型的文件,即,图形样板(*.dwt)、图形(*.dwg)、图形标准(*.dws)。

一般情况下,.dwt文件是标准的样板文件,通常将一些规定的标准性的样板文件设成.dwt文件;.dwg文件是普通的样板文件;而.dws文件是包含标准图层、标注样式、线型和文字样式的样板文件。另外,.dxf文件是用文本形式存储的图形文件,该类型文件能够被其他程序读取,许多第三方应用软件都支持.dxf格式。

如前所述,第一次打开AutoCAD系统就自动创建了一个新文件,对于新建文件来说,创建的方式由STARTUP系统变量确定,当STARTUP变量值为0时,系统将显示如下图所示的“选择样板”对话框,打开该对话框后,系统会自动定位到AutoCAD安装目录的样板文件夹中,用户可以选择使用样板或选择不使用样板来创建新图形。

\begin{figure}[htbp]
\centering
\includegraphics{New_Template.png}
\caption{“选择样板”对话框}
\end{figure}

如果用户要在AutoCAD打开状态下创建新文件,则要通过以下的几种方式实现:

(1)键盘输入New(或QNEW)。

(2)File菜单:在File菜单上单击New子菜单。

(3)Standard工具栏:在Standard工具栏上单击New图标。

(4)快捷键输入【Ctrl+N】。

用上述方法中的任一种命令,AutoCAD都会出现新建文件的对话框。用户可以利用该对话框建立一个新的图形文件“Drawing1.dwg”。

另外,QNEW命令的用法如下:

(1)命令行:QNEW。

(2)工具栏:“标准”$\to$“新建”。

执行上述命令后,系统立即从所选的图形样板创建新图形,而不显示任何对话框或提示。但是,在运行快速创建图形功能之前必须进行如下设置。

(1)将FILEDIA系统变量设置为1,将STARTUP系统变量设置为0。命令行提示如下:

\begin{verbatim}
命令: FILEDIA
输入 FILEDIA 的新值 <1> :
命令: STARTUP
输入 STARTUP 的新值 <0> :
\end{verbatim}

从“工具”$\to$“选项”菜单中选择默认图形样板文件。方法是在“文件”选项卡下单击标记为“样板设置”的节点,然后选择需要的样板文件路径,如下所示:

\begin{figure}[htbp]
\centering
\includegraphics{File_Option.png}
\caption{“选项”对话框的“文件”选项卡}
\end{figure}

而当STARTUP变量值为1时,要新建文件,系统将弹出如下图所示的“创建新图形”对话框。系统提供了从草图开始创建、使用样板创建和使用向导创建三种方式创建新图形。使用样板创建与“选择样板”对话框的样板“打开”类似。

当使用AutoCAD创建一个图形文件时,通常需要先进行图形的一些基本的设置,包括绘图单位、角度、区域等。AutoCAD为用户提供了三种设置方式:

(1)使用样板(Template)

(2)使用缺省设置(Start from Scratch)

(3)使用向导(Wizard)

无论何时开始一张新图,不管是使用向导、样板或缺省创建新图,AutoCAD都将为这张新图命名为“Drawing1.DWG”。这时,可以立即开始在这张新图上绘制图形,并在随后的操作中使用“SAVE”或“SAVE AS”命令将这张新图保存成图形文件。

草图开始创建时,系统提供了如下图所示的英制和公制两种创建方式,这与“选择样板”对话框的“无样板打开-公制”和“无样板打开-英制”类似。

使用向导提供了“高级设置”和“快速设置”两种创建方式,快速设置仅设置单位和区域,在使用向导进行设置时,用户只需要根据向导的提示进行相关设置即可。

\begin{figure}[htbp]
\centering
\includegraphics{New_Draft.png}
\caption{从草图“创建新图形”对话框}
\end{figure}

\chapter{样板文件}


什么是模板?举个例子来讲,假如现在要做一批一模一样的砖块。如果是一个一个地做的话,对于每一个砖块都必须一次一次地考虑每一个砖块是多高、多宽、多厚,砖块的上底面印什么花纹,下底面印什么花纹,然后把砖块逐块地做出来。按这种方法做,效率一定是非常低的。而如果一个专用的砖块模板的话,只需要将泥块放入模板中,就可以一次性做好,而且可以保证统一。

当使用AutoCAD创建一个图形文件时,通常需要先进行图形的一些基本的设置,诸如绘图单位、角度、区域等。我们可以利用系统提供的各种模板来创建新的图形,而在这些模板中已经包含了所有的格式信息。


下面介绍样板的概念与作用。样板文件(Template Files)是一种包含有特定图形设置的图形文件(扩展名为“DWT”),样板文件中通常包含于绘图相关的一些设置,如图层、线型、文字样式等,利用样板图创建新图不仅提高了绘图效率,而且还保证了图形的一致性。通常在样板文件中的设置包括:

(1)单位类型和精度

(2)图形界限

(3)捕捉、栅格和正交设置

(4)图层组织

(5)标题栏、边框和徽标

(6)标注和文字样式

(7)线型和线宽

如果使用样板来创建新的图形,则新的图形继承了样板中的所有设置。这样就避免了大量的重复设置工作,而且也可以保证同一项目中所有图形文件的统一和标准。新的图形文件与所用的样板文件是相对独立的,因此新图形中的修改不会影响样板文件。

AutoCAD中为用户提供了风格多样的样板文件,这些文件都保存在AutoCAD主文件夹的“Template”子文件夹中。

除了使用AutoCAD提供的样板,用户也可以创建自定义样板文件,任何现有图形都可作为样板。如果用户要使用的样板文件没有存储在“Template”文件夹中,则可选择“Browse…(浏览)”打开“Select File(选择文件)”对话框来查找样板文件。

如果用户使用缺省设置创建图形,AutoCAD中包括两种设置方式,分别是“English (feet and inches)(英制)”和“Metric(公制)”,此时通常会基于“acad.dwt”样板文件(以英寸为单位)或“acadiso.dwt”样板(以毫米为单位)来创建图形。

(1)英制:基于英制单位系统和acad.dwt样板创建新图形。缺省区域为12$\times$9英寸。

(2)公制:基于公制单位系统和 acadiso.dwt样板创建新图形。缺省区域为420$\times$297毫米。

如果选择直接进入绘图环境,也可以在后续制图时再调整度量单位等,通过使用units命令调出Drawing units对话框,在其中调整长度和角度等。




\chapter{制作样板文件}

手工设计绘图通常都要在标准大小的图纸上进行。大多数情况下,我们所用的都是印有图框和标题栏的标准图纸,也就是将图纸界线、图框、标题栏等每张图纸上必须具备的内容事先做好,这样既使得图纸规格统一,又节省了绘图者的时间。CAD绘图同样需要这样的准备工作。

有了前面学习的知识,下面我们按1:1比例使用【直线】命令,绘制A4图纸的图框和标题栏。为方便以后作图时无需重复设置,保存为样板图,这样绘图就可以从样板图上开始。

1、开始一张新图

从“选择样板”对话框中选择样板图“acad.dwt”新建一个空白图形文件。

2、图形区域设置

运行【格式】菜单/【图形界限】命令,打开格栅模式,设置297$\times$210的绘图区域。

\begin{verbatim}
指定左下角点或[开(ON)/关(OFF)]<0.0000,0.0000>:(执行默认值);
指定右上角点<420.0000,297.0000>:297,210(设置A4大小的图形界限);
命令:ZOOM;
指定窗口角点,输入比例因子 (nX 或 nXP),或[全部(A)/中心点(C)/动态(D)/
范围(E)/上一个(P)/比例(S)/窗口(W)] <实时>: A(将所设图形界限放至最大);
\end{verbatim}

3、绘图单位和精度设置

选择【格式】菜单/【单位…】命令,屏幕弹出“图形单位”对话框。

(1)在“长度”区内选择单位类型为:“小数”,精度为:“0.0”;

(2)在“角度”区内选择角度类型为:“十进制小数”,精度为: “0”。

(3)在“用于缩放插入内容的单位”列表框中选择图形单位,默认为“毫米”。

4、图层及线型设置

单击【特性】工具栏/按钮,打开“图层管理器”对话框。

单击按钮,新建5个图层,依次起名:粗实线层、中实线、细实线层、虚线层、点划线层。

单击颜色图标方块,在【选择颜色】对话框中,依次为粗实线层、中实线、细实线层、虚线层、点划线层选择不同的颜色,这里索引颜色号依次是1、6、5、4、3。

单击线型图标“Continuous”,弹出【选择线型】对话框,点击【加载】,弹出【加载或重载线型】对话框,选择所需要的线型,虚线“ACADIS002W100”、点划线“ACADIS004W100”。

单击线宽图标“—默认”,弹出【线型】对话框,选择线型的宽度,粗线0.5,中线0.25,细线0.13。

5、绘图框

单击图层工具栏的下拉列表,选择粗实线层为当前层。

单击【绘图】工具栏绘直线按钮,打开正交模式,执行绘直线命令。

\begin{verbatim}
LINE指定第一点:(0,0);
指定下一点或[放弃(U)]: 0,210;
指定下一点或[放弃(U)]:297,210;
指定下一点或[闭合(C)/放弃(U)]297,0;
指定下一点或[闭合(C)/放弃(U)]:C;
命令:【Enter】(重复执行绘直线命令,画内框);@
LINE指定第一点:(25,5);
指定下一点或[放弃(U)]:@ 0,200;@ 267,0;@ 0,200;C;
\end{verbatim}

6.绘制标题栏

将“中实线”设为当前图层,打开对象捕捉模式,单击【绘图】工具栏绘直线按钮,在中实现层上绘标题栏。

\begin{verbatim}
LINE指定第一点:用鼠标拾取图形内框的右下角点;
指定下一点或[放弃(U)]:@ 60, 0;20, 0; 20,0; 20,0;@ 0,8; @ 0,8;
@ 0,16;@ 60,0;15,0;15,0;10,0;20,0;
\end{verbatim}

我们可以用鼠标依次拾取上面绘直线留下的端点和线段中点,绘出其他的线段。

7、修改标题栏外框的图层使其符合线型要求

鼠标选中标题栏外框所有的线段,再点击图层工具栏的下拉列表,选择“粗实线”为当前层,【 Esc】键退出。

8、将图形保存为样板图 

打开“图形另存为”对话框,为设好绘图环境的文件起一个名字:“ A3-1样板图”。名称中“A3”表示图幅大小,“ 1”表示图幅与图形区域的比例为1:1,当建筑绘图的比例为1:100时,我们可以取名为“An-100”。

这样,当我们在画与之适应的新图时,就可以在样板图的基础上开始。需要注意的是,这张样板图适合于创建和编辑二维图形对象中的作图操作,而不适合于工程图形绘制。

通过以上实例,我们会感到单一使用绘直线命令绘图的麻烦和复杂,但是,当我们学习了编辑命令,与绘图命令配合使用之后就会变得非常简单。

在一个项目中,所有新图都可以基于同一相关的样板图建立,既减少了重复设置绘图环境的时间,提高了效率,又可以保证专业或设计项目标准的统一。

\chapter{打开原来的图形}

如果想在原有的图形文件基础上进行有关的操作,就必须打开原有的图形文件。在AutoCAD中,可以通过如下几种方法打开原有的图形文件。

(1)键盘命令:Open。

(2)File菜单:在File菜单上单击Open子菜单。

(3)Standard工具栏 在Standard工具栏上单击Open图标。

(4)快捷键输入【Ctrl+O】。

用上述方式中的任一种命令,AutoCAD将出现Select File(选择文件)对话框。在“查找范围”下拉列表框中选择所要打开的图形文件中,用户既可以在输入框中直接输入文件名打开已有的图形,又可以在文本框中双击要打开的文件名打开已有的图形,下面介绍对话框的内容。

1、单击Open按钮右边的下拉按钮,则可以得到相应的下拉菜单,我们可以从中选出图形的打开方式。

2、Select Initial View:用户若选取Select Initial View复选按钮,则该选项决定的目标图形将以用户定义过的第一个视窗方式打开。

3、View:可以利用View菜单中的子菜单确定Select File对话框中文件的显示方式以及是否预览图形文件。

4、Tools:可以利用Tools菜单中的子菜单来了解该图形文件的信息。单击Find按钮,则可以得到Find对话框。此时可以通过文件的名称和位置来查找文件,也可以通过修改时间来查找文件。

\begin{figure}[htbp]
\centering
\includegraphics{Select_File.png}
\caption{“选择文件”对话框}
\end{figure}

打开一张已经存在的图形文件的操作同新建图形文件的方法一样,输入执行OPEN命令后,打开“选择文件”对话框。与创建新图不同之处在于,在“选择文件”对话框的文件列表框中增加了一类“DXF(*.dxf)”格式的文件可以选择。

用户可以以“打开”、“以只读方式打开”、“局部打开”、“以只读方式局部打开”4种方式打开图形文件,每种方式都对图形文件进行了不同的限制。如果以“打开”和“局部打开”方式打开图形时,可以对图形文件进行编辑。如果以“以只读方式打开”和“以只读方式局部打开”方式打开图形时,则无法对图形文件进行编辑。

利用AutoCAD的局部打开(Partial Open)功能可以有效地提高运行效率。局部打开功能支持空间索引能力,使图纸打开过程更快。局部打开功能提供对于外部参照文件的控制能力,可减少打开大型绘图文件所需要的时间和内存,而绘制和编辑图形的能力则不受影响。

AutoCAD允许同时打开多个AutoCAD文件,用户可以通过Select File对话框打开多个图形;也可以通过在资源管理器中拖放的方式打开多个图形文件,以节省文件打开的时间,提高效率。

在同时打开多个图形文件情况下,可以通过“Ctrl+F6” 或“Ctrl Tab”切换键迅速切换到当前处理图形。

\section{局部打开}

局部打开命令允许用户只处理图形的某一部分,只加载指定视图或图层的几何图形。如果图形文件为局部打开,指定的几何图形和命名对象将被加载到图形文件中。命名对象包括:块(Block)、图层(Layer)、标注样式(Dimension Style)、线型(Linetype)、布局(Layout)、文字样式(Text Style)、视口配置(Viewports)、用户坐标系(UCS)及视图(View)等。

该命令的调用方式同“open”命令。在“Select File(选择文件)”对话框中,用户指定需要打开的图形文件后,单击Open按钮右侧的按钮弹出下拉菜单,选择其中的“Partial Open(局部打开)”或“Partial Open Read-Only(局部打开只读)”项,系统将进一步弹出“Partial Open(局部打开)”对话框,如图所示。

\begin{figure}[htbp]
\centering
\includegraphics{partial_open.png}
\caption{Partial Open对话框}
\end{figure}

在该对话框中,“View geometry to load(要加载几何图形的视图)”栏显示了选定的视图和图形中可用的视图,默认的视图是“Extents(范围)”。用户可在列表中选择某一视图进行加载。

注意:只能加载模型空间视图。若要加载图纸空间几何图形,需要通过加载图纸空间几何图形所在的图层来实现。

在“Layer geometry to load(要加载几何图形的图层)”栏中显示了选定图形文件中所有有效的图层。用户可选择一个或多个图层进行加载,选定图层上的几何图形将被加载到图形中,包括模型空间和图纸空间几何图形。用户可单击Load All按钮选择所有图层,或单击Clear All按钮取消所有的选择。

如果用户选择了“Unload all Xrefs on open(打开时卸载所有外部参照)”开关,则不加载图形中包括的外部参照。

注意:如果用户没有指定任何图层进行加载,那么选定视图中的几何图形也不会被加载,因为其所在的图层没有被加载。

说明:用户也可以使用“partialopen”或“-partialopen”命令以命令行的形式来局部打开图形文件。

\section{局部加载}

对于局部打开的图形,用户还可以通过局部加载将其他未载入的几何图形进行加载,该命令的调用方式为:

(1)菜单:【File(文件)】$\to$【Partial Load(局部加载)】

(2)命令行:partialload

调用该命令后,系统弹出“Partial Load(局部加载)”对话框,如图所示。

\begin{figure}[htbp]
\centering
\includegraphics{partial_load.png}
\caption{Partial Load对话框}
\end{figure}

该对话框的作用与“Partial Open(局部打开)”对话框基本相同,主要区别在于用户在指定视图时,可单击Select Object按钮在绘图窗口中指定一个区域,该区域将成为要加载的视图,并显示在“View geometry to load(要加载几何图形的视图)”列表中。

说明:“partialload”命令具有相应的命令行形式“-partialload”。

\chapter{保存当前的文件图形}

AutoCAD对图形文件进行保存时,若当前的图形文件已经命名,则按此名称保存文件。如果当前图形文件尚未命名,则可以通过“图形另存为”对话框保存已经创建但尚未命名的图形文件。

在AutoCAD中,用户可以利用如下几种方法保存当前的图形文件:

(1)键盘输入Save或Qsave。

(2)File菜单:在File菜单上单击Save或Save as子菜单。

(3)Standard工具栏:在Standard工具栏上单击Save图标。

(4)快捷键输入【Ctrl+S】。

用上述方法中的任一种命令,均出现Save Drawing As(保存图形)对话框。

执行QSAVE命令后,对当前已命名的图形文件直接存盘保存;如该文件尚未命名,则屏幕上弹出“图形另存为”对话框,可从中选择路径并输入文件名,确认后进行保存。AutoCAD还提供了一个SAVE命令,其功能与SAVEAS相似,但只能在命令行中调用。

\begin{figure}[htbp]
\centering
\includegraphics{Save_As.png}
\caption{“图形另存为”对话框}
\end{figure}

用户可利用该对话框,将当前的文件以另外的名字保存,用户若单击Standard工具栏上的保存图标,文件将以当前文件名称进行保存。用户如果不注意,很容易作出用一个文件覆盖另一个文件的错误操作。

用户在绘图的过程中,要记住经常存盘,以免在发生事故(死机、断电)时丢失文件。为了防止因意外操作或计算机系统故障导致正在绘制的图形文件丢失,可以对当前图形文件设置自动保存。操作步骤如下:

(1)利用系统变量SAVEFILEPATH设置所有“自动保存”文件的位置。

(2)利用系统变量SAVEFILE存储“自动保存”文件名。该系统变量存储的文件是只读文件,用户可以从中查询自动保存的文件名。

(3)利用系统变量SAVETIME指定在使用“自动保存”时多长时间保存一次图形。

AutoCAD提供的自动保存功能可以设置具体时间为一分钟到两小时之间的任意时间间隔。可通过【工具】菜单中的【选项】命令,打开“选项”对话框,在“打开和保存”选项卡中设置自动保存的时间,建议每隔10$\sim$30min保存一次绘制的图形,以防止一些意外情况的发生。

另外,在“文件安全措施”选项组中的“临时文件的扩展名”文本框中,可以改变临时文件的扩展名,默认为ac\$。选择“文件”选项卡,在“自动保存文件”选项组中设置自动保存文件的路径,单击“浏览”按钮修改自动保存文件的存储位置。

在“图形另存为”对话框中,“保存于”下拉列表框用于设置图形文件保存的路径,“文件名”文本框用于输入图形文件的名称,“文件类型”下拉列表框用于选择文件保存的格式。在文件格式中,DWG是AutoCAD的图形文件,DWT是AutoCAD的样板文件,这两种格式在AutoCAD中最常用,保存文件的格式会保证了与旧版本文件格式的兼容。

\chapter{设置密码}

在图形管理方面可以使用设置密码的功能。在保存图形时,可以对其进行密码设置,其操作如下:

1、执行保存图形命令后,打开【图形另存为】对话框。

2、单击右上角的【工具】按钮,打开下拉菜单,选择【安全选项】项,系统打开【安全选项】对话框。

\begin{figure}[htbp]
\centering
\includegraphics{Safety.png}
\caption{“安全选项”对话框}
\end{figure}

单击【密码】选项卡,在【用于打开此图形的密码或短语】文本框中输入相应密码。单击【确定】按钮,系统会打开【确认密码】对话框。

当文件设置密码后,需要打开此图形文件时,会打开一个【密码】对话框,要求用户输入密码。如果输入的密码正确,图形打开,否则无法打开图形。

\chapter{关闭图形}

AutoCAD从2000版开始支持多文档环境,因此提供了“close”命令来关闭当前的图形文件,而不影响其他已打开的文件。该命令的调用方式为:

(1)菜单:【File(文件)】$\to$【Close(关闭)】

(2)命令行:close

当用户要退出一个已经修改过的图形文件时,系统会弹出对话框询问是否保存图形。单击“是”按钮,AutoCAD将退出该图形文件并保存用户对该图形文件所作的修改;单击“否”按钮,AutoCAD将退出并不保存所作的修改;单击“取消”按钮,AutoCAD将取消退出。这样用户可以再次确认自己的选择,以免丢失文件。

\chapter{图形修复}

\section{核查命令}


核查命令(AUDIT)是一种诊断工具,用于检查当前的图形数据库,对于每个检查出来的错误,AutoCAD将给出相应说明并提出有关修正办法的建议。该命令的调用方式为:

(1)菜单:【File(文件)】$\to$【Drawing Utilities(绘图实用程序)】$\to$【Audit(核查)】

(2)命令行:audit

调用该命令后,AutoCAD提示用户选择是否更正检测到的任何错误:

\begin{verbatim}
Fix any errors detected? [Yes/No] <N>: 
\end{verbatim}

如果选择“Yes”选项,AutoCAD将自动修复所有检查到的错误,并显示一个检测报告,给出检查到的错误和采取修复措施的详细信息。如果选择“No”选项,则AutoCAD只显示发现错误的报告,而不去修复它们。

如果系统变量AUDITCTL设置为1,AutoCAD将在当前图形所在的目录中创建一个与当前图形同名、扩展名为“.adt”的文本文件,该文件保存了发现的问题和采取的措施。


\section{修复命令}

如果图形含有AUDIT命令不能修复的错误,可以使用RECOVER命令检索图形并修正错误。该命令的调用方式为:

(1)菜单:【File(文件)】$\to$【Drawing Utilities(绘图实用程序)】$\to$【Recover(修复)】

(2)命令行:recover

调用该命令可打开一个图形文件,但AutoCAD将根据图形的表头信息判断正在打开的图形文件是否损坏,如果有损坏的部分则自动修复。

执行完上述命令后,系统打开“图形修复管理器”对话框,打开“备份文件”列表中的文件,可以重新保存,从而进行修复。


\chapter{文件属性}

文件属性是与图形文件相关的信息,包括创建日期、标题、主题、作者、关键词以及注释等,都可以在AutoCAD或Windows资源管理器中访问到。

使用图形文件属性命令可显示和设置当前图形文件的属性信息,用户可以通过如下几种方法访问图形文件属性。

(1)File菜单:在File菜单上单击Drawing Properties…子菜单。

(2)右键菜单:在要访问的文件名上单击鼠标右键。

(3)命令行:dwgprops

用上述方法中的任一种命令,AutoCAD都将弹出Propertie对话框。在图形文件的属性命令的对话框中有General、Summary、Statistics和Custom四个选项卡。

下面分别介绍各选项卡的含义:

\begin{compactenum}
\item General Tab

常规选项卡,显示文件类型、位置、大小、时间和属性等与磁盘文件有关的信息,其显示数据来自操作系统。

\item Summary Tab

摘要选项卡,设置图形标题、主题、作者、关键字、注释和超级链接基地址,其中超级链接基地址是指插入图形中的所有相关链接的基地址。

\item Statistics Tab

统计选项卡,显示文件创建时间、最后修改时间、最后编辑者、修订次数和总编辑时间等信息。这些文件特性是自动维护的,并可利用它们来查找在一个指定的时期内创建或修改的图形。

\item Custom Tab

自定义选项卡,可以利用AutoCAD提供的最多10个自定义域定义有关信息,每个属性包括名称和值两部分。使用自定义属性可以在搜索时便于查找图形。
\end{compactenum}

用户可以把关于图纸的描述信息保存在图形属性中,这样就可以直接利用Windows资源管理器或AutoCAD设计中心的查找工具来检索该图形文件的属性,通过图形属性,就可以在不打开图形的条件下了解该图形文件中所包含的有关信息。

注意:只有在保存图形之后,“图形属性”对话框中输入的属性才能与图形相关联。

\chapter{输出图形}





\section{安装光栅打印机}

在AutoCAD中添加光栅打印机的具体操作步骤如下:

1、在命令提示行输入命令plottermanager,此时AutoCAD会弹出Plotters对话框。

2、单击Add A Plotter Wizard(添加打印机向导)选项,此时AutoCAD会弹出Add Plotter(添加打印机)对话框,单击Next按钮。

3、在弹出的对话框中选择My Computer选项,然后单击Next按钮。此时AutoCAD会弹出Add Plotter对话框。

4、在对话框中,在Manufacturers的列表框中选择Raster File Formats(光栅文件格式)选项,在Models列表框中选择TIFF Version 6(Uncompressed,不压缩)格式。单击Next按钮。

5、在AutoCAD弹出的Add Plotter对话框中单击Next按钮默认系统关于打印机的设置,最后单击Finish按钮结束添加打印机操作。

\section{电子打印}

AutoCAD系统提供了电子打印(ePlot)功能,通过打印发布电子图形,并可发布到Internet上。

\section{批处理打印}

Autodesk公司在AutoCAD系统中附带了一个批处理打印程序(Batch Plot Utility),用于打印一系列AutoCAD图形。用户可以立刻打印这些图形,也可以将它们保存在批处理打印文件(BP3)中以供将来使用。

批处理打印实用程序独立于AutoCAD运行,用户可在操作系统中的AutoCAD程序组里启动该程序。
通常在执行批处理打印时,不能访问与批处理打印关联的AutoCAD窗口。

\section{图形输出}

可以通过如下操作来设置在Word文档中插入CAD图形的线宽。

打开要插入的CAD图形,按Ctrl+P组合键,此时会弹出Plot(打印)对话框。

选择该对话框中的Plot Device(打印机设备)选项卡,在Plotter configuration(打印机配置)的下拉选项中选取“TIFF Version 6(Uncompressed).pc3”,然后单击右边的Properties(属性)按钮,此时会弹出Plotter Configuration Editor对话框。

选取Custom Properties(自定义特征)选项,然后再单击Custom Properties(自定义特征)按钮,在AutoCAD弹出的对话框中设置背景颜色,然后返回Plot对话框。

在Plot对话框中选取Plot Setting(打印设置)选项卡,利用该选项卡的界面可以设置输出的图形的大小、输出图形的比例以及其他的一些设置。最后再将图形文件插入到Word文档中即可。在AutoCAD中,我们可以通过图形设置图形的颜色来设置输出的线宽。



\section{改变文件输出格式}

在与其他格式的图形进行数据交换时,AutoCAD可以对几种不同的图形格式进行转换,以便用户更方便地共享和使用图形数据。

在AutoCAD中,可以通过如下几种方法将任意图形以其它文件格式保存:

1、在AutoCAD中选取需要以其它格式保存的图形。

2、单击File菜单中的Export子菜单或运行export(或别名exp)命令,此时AutoCAD会弹出Export Data对话框。

3、从Files of type(文件类型)的下拉选项中选取自己所需要的文件格式,然后单击Save按钮即可。

AutoCAD中提供.WMF、.SAT、.STL、.DXX、.BMP、.3DS以及.DWG几种文件格式。其中:

\begin{compactitem}
\item .WMF(Windows图元文件)格式是AutoCAD文件的默认格式,在Word文档中插入.WMF文件后,可以通过双击插入的图片进行编辑。但是利用.WMF格式保存的AutoCAD图形在Word中无法打印出图形的线宽。

\item .SAT(ACIS实体对象文件)就是以ACIS文件格式保存利用AutoCAD绘制的三维实体图形。

\item .STL(实体对象立体印刷文件)就是以STL格式保存三维实体图形。

\item .BMP(独立于设备的位图文件)格式就是将AutoCAD绘制的图形保存为图片的格式,在Word中插入.BMP格式时,AutoCAD图形可以表示出线的宽度,但是对于分辨率要求较高的图形而言就不适合。

\item .3DS格式就是将AutoCAD绘制的图形保存为三维动画图形的格式。

\item .EPS(封装PostScript文件)

\item .DXX(属性提取DXF文件)
\end{compactitem}

对于每一种文件格式的输出,AutoCAD也提供了相应的专用命令,如表所示。

\begin{table}[htbp]
\centering
\caption{文件输出命令}
\begin{tabular}{|l|l|}
\hline
命令		& 作用			\\
\hline
WMFOUT	& 以WMF格式输出	\\
\hline
ACISOUT	& 以SAT格式输出		\\
\hline
STLOUT		& 以STL格式输出		\\
\hline
PSOUT		& 以EPS格式输出		\\
\hline
ATTEXT		& 以DXX格式输出	\\
\hline
BMPOUT	& 以BMP格式输出	\\
\hline
3DSOUT	& 以3DS格式输出	\\
\hline
\end{tabular}
\end{table}

下面详细介绍如何在Word中插入以.WMF格式保存的AutoCAD图形:

1、打开要插入图形的Word文档,然后将光标移到要插入AutoCAD图形的位置。

2、单击Insert(插入)菜单中的Object(对象)子菜单,此时Word会弹出Object对话框。

3、在Object对话框中选取Create from File选项卡,选中Link to File复选项,然后单击Browse按钮,从Browse对话框中选取需要插入的图形文件,然后单击Insert按钮,再单击Object对话框中的OK按钮,此时可以得到相应的Word文档。

4、双击Word文档中的图形就可以打开AutoCAD图形,此时就可以对以.WMF格式保存的文件进行修改。当然也可以直接在Word中调整整个图形文件的大小。

另外,也可以通过Save Image命令将AutoCAD绘制的图形保存为其它格式。具体的操作步骤如下:

1、在“Command:”提示行输入命令saveimg,此时AutoCAD会弹出Save Image对话框。

2、可以利用Save Image对话框来确定AutoCAD文件的具体格式以及图形区域。

3、执行完以上操作后,单击OK按钮即可。

同样的,AutoCAD也可以打开并使用其他格式的图形文件,并可进行格式转换。其调用方式为:

(1)工具栏:【Insert(插入)】$\to$插入图形文件

(2)命令行:import(或别名imp)

调用该命令后,AutoCAD将弹出“Import File(输入文件)”对话框,用户可选择WMF、SAT和3DS格式的文件。

此外,用户可以利用菜单和专用命令来输入指定格式的文件,如表所示。

\begin{table}[htbp]
\centering
\caption{文件输入命令}
\begin{tabular}{|l|l|l|}
\hline
“Insert(插入)”菜单	&  命令			& 作用			\\
\hline
3D Studio…			& 3DSIN		& 输入3DS格式文件	\\
\hline
ACIS File…			& ACISIN		& 输入SAT格式文件	\\
\hline
Drawing Exchange Binary…	& DXBIN	& 输入DXB格式文件	\\
\hline
Windows Metafile…	& WMFIN		& 输入WMF格式文件	\\
\hline
Markup…			& RMLIN		& 输入RML格式文件	\\
\hline
\end{tabular}
\end{table}

\chapter{绘图空间}

AutoCAD有两个绘图空间,即模型空间(Model space)和图纸空间(Paper space)。用户的大多数绘图和设计是在模型空间进行的。

在模型空间,用户根据所建立的模型,按一定尺寸完成模型的造型,同时也可根据需要用多个视图表示模型,标上必要的尺寸标注和文本则完成所需的绘图。

模型空间是一个三维坐标空间,主要用于几何模型的构建。而在对几何模型进行打印输出时,则通常在图纸空间中完成。

图纸空间就象一张图纸,打印之前可以在上面排放图形。图纸空间用于创建最终的打印布局(Layout),而不用于绘图或设计工作。


在图纸空间,同样允许用户完成类似模型空间的工作。同时,视窗数目和位置不受限制,用户可把视窗作为AutoCAD 的对象进行诸如:移动( Move)、拷贝(Copy)等编辑操作。

模型空间和图纸空间可相互切换,切换时只需单击状态栏上的Model或Paper切换按钮即可。

同一个图形文件,在图纸空间的显示如图1所示,在模型空间的显示如图2所示。

\begin{figure}[htbp]
\centering
\includegraphics{layout_sample.png}
\caption{图纸空间}
\end{figure}

\begin{figure}[htbp]
\centering
\includegraphics{model_sample.png}
\caption{模型空间}
\end{figure}


AutoCAD有两个不同的空间:即模型空间和图纸空间(通过使用LAYOUT标签)。
 
模型空间中视口的特征:

1、在模型空间中,可以绘制全比例的二维图形和三维模型,并带有尺寸标注。

2、模型空间中,每个视口都包含对象的一个视图。例如:设置不同的视口会得到俯视图、正视图、侧视图和立体图等。

3、用VPORTS命令创建视口和视口设置,并可以保存起来,以备后用。

4、视口是平铺的,它们不能重叠,总是彼此相邻。

5、在某一时刻只有一个视口处于激活状态,十字光标只能出现在一个视口中,并且也只能编辑该活动的视口(平移、缩放等)。

6、只能打印活动的视口。如果UCS图标设置为ON,该图标就会出现在每个视口中。

7、系统变量MAXACTVP决定了视口的范围是2到64。

图纸空间中视口的特征:

1、状态栏上的PAPER取代了MODEL。

2、VPORTS、PS、MS、和VPLAYER命令处于激活状态。(只有激活了MS命令后,才可使用PLAN、VPOINT和DVIEW命令)。

3、视口的边界是实体,可以删除、移动、缩放、拉伸视口。

4、视口的形状没有限制。例如:可以创建圆形视口、多边形视口等。

5、视口不是平铺的,可以用各种方法将它们重叠、分离。

6、每个视口都在创建它的图层上,视口边界与层的颜色相同,但边界的线型总是实线。出图时如不想打印视口, 可将其单独置于一图层上,冻结即可。

7、可以同时打印多个视口。

8、十字光标可以不断延伸,穿过整个图形屏幕,与每个视口无关。

9、可以通过MVIEW命令打开或关闭视口,SOLVIEW命令创建视口或者用VPORTS命令恢复在模型空间中保存的视口。在缺省状态下,视口创建后都处于激活状态。关闭一些视口可以提高重绘速度。

10、在打印图形且需要隐藏三维图形的隐藏线时,可以使用MVIEW命令$\to$HIDEPLOT拾取要隐藏的视口边界即可。

11、系统变量MAXACTVP决定了活动状态下的视口数是64。

切记:当我们第一次进入图纸空间时,看不见视口,必须用VPORTS或MVIEW命令创建新视口或者恢复已有的视口配置(一般在模型空间保存)。可以利用MS和PS命令在模型空间和LAYOUT(图纸空间)中来回切换。 


\chapter{图纸空间}

在AutoCAD中,图纸空间(Paper Space)是以布局的形式来使用的。一个图形文件可包含多个布局,每个布局代表一张单独的打印输出图纸。在绘图区域底部选择布局(Layout)选项卡,就能查看相应的布局。通过点击不同的布局选项卡,就可以进入相应的图纸空间环境。

图纸空间主要的作用是用来出图的,就是把我们在模型空间绘制的图,在图纸空间进行调整、排版,因此这个过程称为“布局(Layout)”是非常恰当的。


\begin{figure}[htbp]
\centering
\includegraphics{layout_space.png}
\caption{图纸空间}
\end{figure}


在AutoCAD中,通过布局(Layout)和打印(Plot)来完成图形输出,是绘制图纸并提交方案的最终目的。其中,布局是一种图纸空间环境,它模拟图纸页面,提供直观的打印设置。

图纸空间可以理解为覆盖在模型空间上的一层不透明的纸,若要从图纸空间看模型空间的内容,就必须进行开“视口”操作,也就是“开窗”。

图纸空间是一个二维空间,也就是在图纸空间绘制的对象虽然也有$z$坐标,但是三维操作的一些相关命令在图纸空间不能使用,导致它所显示的特性跟二维空间相似。


在布局中可以创建并放置视口对象,还可以添加标题栏或其他几何图形。可以在图形中创建多个布局以显示不同视图,每个布局可以包含不同的打印比例和图纸尺寸。布局显示的图形与图纸页面上打印出来的图形完全一样。

“视口”可以想象成是在图纸空间这张“纸”上开的一个口子,这个口子的大小、形状可以随意使用(详见视图菜单下的视口项)。在视口里面对模型空间的图形进行缩放(ZOOM)、平移(PAN)、改变坐标系(UCS)等的操作,可以理解为拿着这张开有窗口的“纸”放在眼前,然后离模型空间的对象远或者近(等效ZOOM)、左右移动(等效PAN)、旋转(等效UCS)等操作,更形象的说,就是指这些操作是针对图纸空间这张“纸”的,这就可以理解为什么在图纸空间进行若干操作,但是对模型空间没有影响的原因。如果不再希望改变布局,可以“锁定视口”。

注意,使用诸如STRETCH、TRIM、MOVE、COPY等编辑命令对对象所作的修改,等效于直接在模型空间修改对象,有时为了使单张图纸的布局更加紧凑、美观就需要从图纸空间进入模型空间,进行适当的编辑操作。 


在图纸空间中,用户可随时选择“模型(Model)”选项卡(或命令行输入model)来返回模型空间,也可以在当前布局中创建浮动视口来访问模型空间。浮动视口相当于模型空间中的视图对象,用户可以在浮动视口中处理模型空间对象。在模型空间中的所有修改都将反映到所有图纸空间视口中。

\begin{compactitem}
\item 用户可在布局中的浮动视口上双击鼠标左键,进入视口中的模型空间,或者是从状态栏中选择“模型”按钮,也可以使用MS命令从图纸空间进入模型空间。而如果在浮动视口外的布局区域双击鼠标左键,则回到图纸空间,相当于从状态栏中选择“图纸”按钮。


在图纸空间中,可以使用缩放、平移等手段使我们需要打印的那部分显示在图框范围内(当然一般情况下,使用的是已经做好的图,一般是图的大致轮廓,如果知道最终的出图比例当然可以最后再做,但是新建的图一般都不知道)。

\item 用户可在视口范围以外或用PS命令双击鼠标,就从模型空间退回图纸空间。如果不知道出图比例,就需要试算一下,这个过程很简单,先选择视口,如果直接选不中的话,可以使用框选(从左往右框)。


然后点击特性窗左上角,选中“视口(1)”,表示选中的对象中只有一个视口。这时我们可以查看并修改视口的特性了,那么主要就是用来设定出图的比例了。


在视口的属性中我们需要关心的是两个地方,一个使“显示锁定”,一个是“自定义比例”,其中“显示锁定”用来将最终设置好比例的视口锁定,就不会因误操作导致重复调整视口了。

“自定义比例”就是比例系数。图中显示的是0.015,在模型空间的绘图单位是厘米,图框为A3(420$\times$297)没有进行比例缩放操作,就是说这里出图是以毫米为单位的。

用“AutoCAD绘图用系数”工具计算,在“比例系数”后输入0.015回车后如下图显示,得到比例尺$1:666.66667$,这显然不是我们想要的,这时对得到的比例取整取$1:600$,输入到“比例尺”内,程序适时算出“比例系数”为0.0166666,将这个数输入到AutoCAD特性窗中的“自定义比例”回车确认以后,出图的比例就设定为$1:600$了,同时程序告诉我们,全局比例为60,将这个数输入到“标注样式”中“调整”标签下“使用全局比例”中即可完成对标注样式的设定,如果最终出图文字高度为3毫米,“Mode字高”告诉你在模型空间写的单行或多行文字高度应该为180,出图字高为4毫米,则模型空间高度应该为$4\times 60=240$毫米。如果某条多义线出图宽度想设定为0.8毫米,则模型空间应该设定这条多义线的宽度为$0.8\times 60=48$毫米。

如果知道出图比例为$1:80$,模型空间绘图单位为毫米,出图也为毫米,则计算出比例系数0.0125,全局比例为80,依照上面介绍的就可以确定。

 
至此,布局的基本操作已经完成。


\end{compactitem}



\begin{figure}[htbp]
\centering
\includegraphics{model_space.png}
\caption{在布局中访问模型空间}
\end{figure}

熟练的使用图纸空间,需要配合几个方面的设置,这可能要改变自己以前绘图以及出图的习惯,不过用这些去换来轻松的操作,是完全值得的。首先对在模型空间绘图有以下几个说明或者要求,如果做不到这几条,图纸空间对使用者来说依然是混乱的。

\begin{compactenum}
\item 最好严格按照$1:1$的方式绘图。

这样不仅作图时方便,以后修改也方便,重要的是在使用图纸空间出图时更加灵活方便。
\item 明确自己在模型空间绘图所使用的单位。

比如用毫米为单位,那么1米就要用1000个单位,用厘米为单位,那么1米就要用100个单位,需要说明的是,在AutoCAD中设定的所谓的“绘图单位”是没有意义的,绘图的单位应该是在使用者心中,这也是AutoCAD灵活的一个方面,因为这样在AutoCAD中绘制一条长度为1的线段,可以代表任何一个单位长度。
\item 设置标准图框。

每个工程在进行之前,就应该作一个标准图框供参与该项目的所有人员使用,从而保证出图风格的一致。如常用的A3图框,在模型空间做好,并以毫米为单位。左下角的坐标为(0,0)为便于使用者插入图框,应单独存为一个DWG文件。
\item 设置打印比例。

打印时其他的设置基本适合各自的习惯即可,现在唯一要求不同的是要求按$1:1$打印或者$1:1.02$(就是缩小至98\%)打印,这样的好处很明显,我们不需要进行繁琐的比例换算工作,所有的比例问题在图纸空间和标注样式里面设置,最终同一个工作组的绘图风格基本一致。
\item 标注样式的设置

打开“标注样式管理器”,在“样式”列表中列出当前文档中已经存在的标注样式,一般在作图中可能会用到1$\sim$2种样式,最好把名称改为自己常用的,比如“GZ40QH\_120”表示国主40项目桥涵组比例尺为$1:120$。

最好是在设置完一个标注样式以后再以此样式为基础样式新建其他的样式,按照这种方法,“GZ40QH\_120”与“GZ40QH\_250”两种标注样式只有一个参数不同,就是“调整”标签下的“使用全局比例”分别为$12$和$25$(前提是模型空间绘图单位为“cm”),设置非常简单。样式设置好了以后,还可以给同事共享,他们就不需要进行繁琐的设置工作了。

我们对“GZ40QH\_120”进行修改,首先对诸如箭头大小、超出尺寸线长度、文字高度等进行设置,原则是最终要打印成什么样子就设置成什么样子,比如最终打印出来箭头大小为2.0mm,文字高为3.0mm,超出尺寸线1.5mm,那么在设置这些值时就设置成前面的数值,先不管它的放大系数。然后切换到“调整”标签,选择“标注特征比例”框架下的“使用全局比例”,在其后的文本框内输入后面计算得到的全局比例(比如12或25)。这样设置完以后,确定回到标注样式管理器,将需要用的标注样式置为当前,这样在模型空间进行标注时,其大小就比较合适,在出图时也不会出现混乱的图幅了。这样设置一个明显的优点是在调整标注样式时,我们基本上可以不管出图比例是多少,直接将各项设置成最终出图的效果,需要针对不同出图比例调整的唯一一个参数就是全局比例。
\item 在做完一副图的大致轮廓,也就是在没有进行大量的文字及尺寸标注之前就要按以下方法确定最终的出图比例,也就是需要确定全局比例、自定义比例系数、模型空间文字大小(同时可以确定多义线的宽度)。

(1)图的大致轮廓绘制好了以后,切换到图纸空间(布局)。一般切换以后就会提示布局设置,而且每次都会提示。

(2)插入标准图框,使用MV(MVIEW)命令沿内边框开视口。
\end{compactenum}

一般情况下,设计布局环境包含以下几个步骤:

\begin{compactenum}
\item 创建模型图形。
\item 配置打印设备。
\item 激活或创建布局。
\item 指定布局页面设置,如打印设备、图纸尺寸、打印区域、打印比例和图形方向。
\item 插入标题栏。
\item 创建浮动视口并将其置于布局。
\item 设置浮动视口的视图比例。
\item 按照需要在布局中创建注释和几何图形。
\item 打印布局。
\end{compactenum}

在同一张图纸上需要用到两种以上的比例出图时,其实就是需要用到两套标注样式和两个视口,对应图形部分的文字高度需要分别设置,用于$1:600$的图形部分字高为180,标注样式用GZ40QH-600,用于$1:120$的图形部分字高为36,标注样式用GZ40QH-120。


如果一张图中有多个视口,打印时视口线也会打印出来,解决的办法是把这部分视口放置在“DefPoints”图层中,或者是“定义点”图层,放置在这个图层中的对象是不会被打印出来的,如果某个图层中的东西也不想打印出来,可以在图层管理器中点击对应图层后面的那个打印机,使它出现红“$\times$”。

再有,有时在模型空间设置的线型为虚线,在图纸空间显示的却为实线,这是线形比例问题,首先要从图纸空间进入模型空间,可以尝试用“RE”命令刷新一下,如果还没有正常,就选择那条虚线(当然也可以是点画线),调整它的“线型比例”,有时一次调整不好,就需要多调整几次,达到要求为止。


调整“线型比例”时,因为使用不同的线性所以会导致比例值差异很大,所以建议虚线只用“HIDDEN”,点画线只用“CENTER”,这两种的“线型比例”可以跟全局比例联系起来。

如果“线型比例”过大过小在模型空间都会显示不出线型来,所以建议虚线要用一种或两种颜色来区分。 

最后,在对页面设置完成以后还要进行最后一项工作,就是“页面设置”不要以为这个工作无所谓,设置好以后还可以使用AutoCAD的辅助工具进行批量打印,不过这要求一个文件只有一个布局,一个布局里面只有一张图,并且这张图的打印设置已经设置完成。鼠标右单击“布局1”在快捷菜单中选择“页面设置”。


设置界面跟打印设置几乎一样,如果这次的打印需要与上次打印设置基本一样,直接选<上一次打印>即可完成大部分设置工作,唯一要做的就是点击“窗口”选择打印范围,然后确定,最后预览一下,可以到所有工作完成以后,统一一次打印出来。

\chapter{页面设置}

页面设置就是随布局一起保存的打印设置。指定布局的页面设置时,可以保存并命名某个布局的页面设置,然后将命名的页面设置应用到其他布局中。

在绘图任务中首次选择布局选项卡时,将显示单一视口,并以带有边界的表来标示当前配置的打印机的纸张大小和图纸的可打印区域。AutoCAD显示“页面设置”对话框,从中可以指定布局和打印设备的设置。指定的布局设置将随布局一起保存。

用户也可调用页面设置命令来访问“Page Setup(页面设置)”对话框,调用方式为:

(1)工具栏:“Layouts(布局)”$\to$页面设置图标

(2)菜单:【File(文件)】$\to$【Page Setup…(页面设置)】

(3)快捷菜单:在布局选项卡上单击右键,选择“Page Setup…(页面设置)”项

(4)命令行:pagesetup

调用该命令后,系统弹出“Page Setup(页面设置)”对话框,如图所示。

\begin{figure}[htbp]
\centering
\includegraphics{page_setup_layout.png}
\caption{Page Setup对话框}
\end{figure}


在该对话框中各项设置如下:

1、“Layout name(布局名)”编辑框:显示并修改当前布局的名称。

2、“Page setup name(页面设置名)”下拉列表:命名并保存当前的页面设置。

如果单击按钮可弹出“用户定义的页面设置”对话框。在该对话框中可保存和管理所有用户定义的页面设置。其中,用户单击按钮来输入其他图形中的页面设置。

3、“Plot Device(打印设备)”选项卡:设置打印设备。包括:

\begin{compactitem}
\item “Plotter configuration(打印机配置)”:选择和配置需要使用的打印设备。
\item “Plot style table(打印样式表)”:选择或创建打印样式表。
\item Options按钮:单击该按钮可显示“Options(选项)”对话框的“Plotting(打印)”选项卡,用于查看或修改当前布局及其应用的打印样式设置。
\end{compactitem}

4、“Layout Settings(布局设置)”选项卡:在其中可进行如下设置。

\begin{compactitem}
\item “Paper size and paper units(图纸尺寸和图纸单位)”:显示选定的打印设备,并在“Paper size(图纸尺寸)”下拉列表中给出了打印设备可用的标准图纸尺寸。如果没有选定打印机,则显示全部标准图纸尺寸。

此外,用户还可以指定以“Inches(英寸)”还是“mm(毫米)”为单位进行打印。

\item “Drawing orientation(图形方向)”:选择“Portrait(纵向)”表示用图纸的短边作为图形页面的顶部。“Landscape(横向)”则表示图纸的长边作为图形页面的顶部。无论使用哪一种图形方式,都可以通过选择“Plot upside-down(反向打印)”开关来得到相反的打印效果。

\item “Plot area(打印区域)”:指定要打印的区域,可选择以下5种定义中的一种:

\begin{compactenum}
\item “Layout(布局)”:打印指定图纸尺寸页边距内的所有对象。
\item “Extents(范围)”:打印图形的当前空间中的所有几何图形。
\item “Display(显示)”:打印“模型”选项卡的当前视口中的视图。
\item “View(视图)”:打印一个已命名视图。如果没有已命名视图,此项不可用。
\item “Window(窗口)”:打印由用户指定的区域内的图形。用户可单击Window按钮返回绘图区来指定打印区域的两个角点。
\end{compactenum}

\item “Plot scale(打印比例)”:选择或定义打印单位(英寸或毫米)与图形单位之间的比例关系。如果选择了“Scale lineweights(缩放线宽)”项,则线宽的缩放比例与打印比例成正比。

\item “Plot offset(打印偏移)”:指定相对于可打印区域左下角的偏移量。如选择“Center the plot(居中打印)”,则自动计算偏移值,以便居中打印。

\item “Plot options(打印选项)”:选择各项可具有如下作用:

\begin{compactenum}
\item “Plot object lineweight(打印对象线宽)”:打印线宽。
\item “Plot with plot style(打印样式)”:按照对象使用的和打印样式表中定义的打印样式进行打印。
\item “Plot paperspace last(最后打印图纸空间)”:先打印模型空间的几何图形,然后再打印图纸空间的几何图形。
\item “Hide objects(隐藏对象)”:打印在布局环境(图纸空间)中删除了对象隐藏线的布局。
\end{compactenum}

\end{compactitem}

完成以上设置后,用户可直接单击Plot按钮来进行打印。

如果不希望每次新建图形布局时都出现“页面设置”对话框,可取消对话框中的“Display when creating a new layout”选项。

如果PAPERUPDATE系统变量设置为1,将提示选定的打印设备是否支持布局中现有的图纸尺寸。图纸尺寸将自动更新以反映选定打印设备的缺省图纸尺寸。

\chapter{布局设置}

在图纸空间中利用布局来进行打印设置,主要包括布局的创建及其打印设置。同时,在布局中创建和使用浮动视口,可以在绘图中布局与模型中进行相关的切换。


\section{使用向导创建新布局}


布局向导用于引导用户来创建一个新的布局,每个向导页面都将提示用户为正在创建的新布局指定不同的版面和打印设置。调用布局向导命令的方式为:

(1)菜单:【Tools(工具)】$\to$【Wizards(向导)】$\to$【Create Layout…(创建布局)】

(2)命令行:layoutwizard

调用该命令后,AutoCAD将显示“Create Layout(创建向导)”对话框,如图所示。

\begin{figure}[htnp]
\centering
\includegraphics{create_layout.png}
\caption{Create Layout对话框}
\end{figure}

下面依次对各个步骤进行介绍:

1、“Begin(开始)”:指定新建布局的名称。

2、“Printer(打印机)”:选择已匹配的打印机。

3、“Paper Size(图纸尺寸)”:选择图纸尺寸、图纸单位。

4、“Orientation(图纸方向)”:选择图纸的打印方向。

5、“Title Block(标题块)”:列表中显示了AutoCAD所提供的样板文件中的标准标题栏,包括多种ANSI(美国国家标准化协会)和ISO(国际标准化组织)标题栏。其中ANSI标题栏是以英寸为单位绘制的,而ISO、DIN和JIS标题栏则是以毫米为单位绘制的。如果需要,用户可选择其中一种并以“Block(块)”或“Xref(外部参照)”的形式插入到当前图形文件中。

\begin{figure}[htbp]
\centering
\includegraphics{create_layout_title_block.png}
\caption{Title Block步骤}
\end{figure}

6、“Define(定义)”:用户可指定视口的形式和比例,可供选择的视口形式有如下四种:

\begin{compactitem}
\item “None(无)”:不创建视口。
\item “Single(单个)”:创建单一视口。
\item “Std. 3D Engineering Viewts(标准工程视图)”:创建工程图中常用的标准三向视口。标准的三维工程配置是包括俯视图、主视图、侧视图和等轴测视图在内的2$\times$2阵列。
\item “Array(阵列)”:创建指定数目的视口,这些视口排列为矩形阵列。
\end{compactitem}

\begin{figure}[htbp]
\centering
\includegraphics{create_layout_define_viewports.png}
\caption{Define步骤}
\end{figure}

7、“Pick Location(定位)”:指定视口在图纸空间中的位置。

8、“Finish(结束)”:结束向导命令,并根据以上设置创建新布局。

用向导完成布局设置之后,用户也可随时调用“Page Setup(页面设置)”对话框来修改新布局的任何设置。


\section{使用布局命令}

AutoCAD中的布局命令可实现布局的创建、删除、复制、保存和重命名等各种操作。调用该命令的方式为:

(1)命令行:-layout(或别名lo)、layout

调用该命令后,系统将显示如下各个选项:

\begin{verbatim}
Enter layout option [Copy/Delete/New/Template/Rename/SAveas/Set/?]
\end{verbatim}

各项的作用如下为:

(1)“Copy(复制)”:复制指定的布局,复制后的新的布局选项卡将插到被复制的布局选项卡之后。选择该项后系统将提示用户指定用于复制的布局名称和复制后新的布局名称:

\begin{verbatim}
Enter name of layout to copy <Layout1>:
Enter layout name for copy <Layout1 (2)>:
Layout "Layout1" copied to "Layout1 (2)".	
\end{verbatim}

(2)“Delete(删除)”:删除指定的布局。选择该项后系统将提示用户指定要删除的布局名称:

\begin{verbatim}
Enter name of layout to delete <Layout1>:
Layout "Layout1" deleted.
\end{verbatim}


(3)“New(新建)”:使用指定的名称和缺省的打印设备来创建一个新的布局。选择该项后系统将提示用户指定创建的布局名称: 

\begin{verbatim}
Enter new Layout name <Layout3>:
\end{verbatim}

(4)“Template(样板)”:插入样板文件中的布局。选择该项后,系统将弹出“Select Template From File(从文件中选择样板)”对话框,用户可在AutoCAD系统主目录中的“Template”子目录中选择AutoCAD所提供的样板文件,也可以使用其他图形文件(包括DWG文件和DXF文件)。

当用户选择了某一文件后,将弹出“Insert Layout(s)(插入布局)”对话框,该对话框中显示了该文件中的全部布局,用户可选择其中一种或几种布局(包括所有几何图形)插入到当前图形文件中。


(5)“Rename(重命名)”:给指定的布局重新命名。选择该项后系统将提示用户指定需要改名的布局和新的布局名称:

\begin{verbatim}
Enter layout to rename <Layout1>:
Enter new layout name:
\end{verbatim}

布局名最多可以包含255个字符,不区分大小写。布局选项卡中只显示最前面的32个字符。注意,布局名必须唯一。

(6)“SAveas(另存为)”:保存指定的布局。选择该项后系统将提示用户指定需要保存的布局名称:

\begin{verbatim}
Enter layout to save to template <Layout1>:
\end{verbatim}

然后系统将弹出“Create Drawing File(创建图形文件)”对话框来指定保存的图形文件名称和路径。用户可将指定布局保存为DWT、DWG和DXF等格式。


(7)“Set(设置):设置指定布局为当前布局。选择该项后系统将提示用户指定布局名称,并将其设置为当前布局:

\begin{verbatim}
Enter layout to make current <Layout1>:
\end{verbatim}

(8)“?”:列出图形中定义的所有布局。


\section{使用布局的其他方式}

使用“Layouts(布局)”工具栏,可以通过点击图标来调用“layout”命令中的“New”选项来新建布局,也可以通过点击图标调用“layout”命令中的“Template”选项来通过样板插入布局。

用户也可在布局选项卡中单击右键,会弹出快捷菜单。利用该快捷菜单用户还可以实现“Move(移动)”和“Select All Layouts(选择所有布局)”等功能。

\chapter{浮动视口}

在模型空间中可以创建平铺视口。同样,在图纸空间中也可以创建视口,称为浮动视口。与平铺视口不同,浮动视口可以重叠,或进行编辑。

在构造布局时,可以将视口视为模型空间中的视图对象,对它进行移动和调整大小。浮动视口可以相互重叠或者分离。因为浮动视口是AutoCAD对象,所以在图纸空间中排放布局时不能编辑模型。要编辑模型必须切换到模型空间。

将布局中的视口设为当前后,就可以在浮动视口中处理模型空间对象。在模型空间中的所有修改都将反映到所有图纸空间视口中。

使用浮动视口的好处之一是:可以在每个视口中选择性地冻结图层。冻结图层后,就可以查看每个浮动视口中的不同几何对象。通过在视口中平移和缩放,还可以指定显示不同的视图。

\section{创建浮动视口}

在布局中,用户可调用“vports”命令弹出“Viewports(视口)”对话框来创建一个或多个矩形浮动视口,如同在模型空间中创建平铺视口一样


此外,在图纸空间中还可创建各种非矩形视口。用户可采用如下两种方式进行创建:

\subsection{创建多边形视口}

(1)工具栏:“Viewports(视图)”$\to$多边形视口

(2)菜单:【View(视图)】$\to$【Viewports(视口)】$\to$【Polygonal Viewport(多边形视口)】

系统将提示用户指定一系列的点来定义一个多边形的边界,并以此创建一个多边形的浮动视口。


\subsection{从对象创建视口}

(1)工具栏:“Viewports(视图)”$\to$对象

(2)选择菜单【View(视图)】$\to$【Viewports(视口)】$\to$【Object(对象)】

系统将提示用户指定一个在图纸空间绘制的对象,并将其转换为视口对象。

下图显示了非矩形视口的一个示例。

\begin{figure}[htbp]
\centering
\includegraphics{polygonal_viewport.png}
\caption{非矩形视口}
\end{figure}

与模型空间不同,在图纸空间中“vports”命令的命令行形式“-vports”提供了更多的功能。调用该命令后,系统将提示如下:

\begin{verbatim}
Command: -vports
Specify corner of viewport or 
[ON/OFF/Fit/Hideplot/Lock/Object/Polygonal/Restore/2/3/4] <Fit>:
\end{verbatim}

各项说明如下: 

\begin{compactitem}
\item 用户可直接指定两个角点来创建一个矩形视口。
\item “ON(开)”:打开指定的视口,将其激活并使它的对象可见。
\item “OFF(关)”:关闭指定的视口。如果关闭视口,则不显示其中的对象,也不能将其置为当前。
\item “Fit(布满)”:创建充满整个显示区域的视口。视口的实际大小由图纸空间视图的尺寸决定。
\item “Hideplot(消隐出图)”:从图纸空间(布局)打印时,删除视口中隐藏线。
\item “Lock(锁定)”:锁定当前视口,与图层锁定类似。锁定视口后,在用“zoom”命令放大图形时,不会改变视口的比例。
\item “Object(对象)”:将图纸空间中指定的对象换成视口。
\item “Polygonal(多边形)”:指定一系列的点创建不规则形状的视口。
\item “Restore(恢复)”:恢复保存的视口配置。
\item “2”:将当前视口拆分为两个视口,与在模型空间中用法类似。
\item “3”:将当前视口拆分为三个视口,与在模型空间中用法类似。
\item “4”:将当前视口拆分为大小相同的四个视口。
\end{compactitem}


注意:不能保存和命名在布局中创建的视口配置,但可以恢复在模型空间中保存的视口配置。



\section{修改视口对象}

在图纸空间中,视口也是图形对象,因此具有对象的特性,如颜色、图层、线型、线型比例、线宽和打印样式等。用户可以使用AutoCAD任何一个修改命令对视口进行操作,如MOVE、COPY、STRETCH、SCALE和ERASE等,也可以利用视口的夹点和特性进行修改。

注意:只有在图纸空间中才能创建和操作浮动视口,但是在浮动视口中不能编辑模型。

如果冻结非矩形视口的边界图层,将不显示边界,也不剪裁视口。如果关闭边界图层而不是冻结它,视口仍会被剪裁。


\section{使用浮动视口}


\subsection{通过视口访问模型空间}


在布局中工作时,在图纸空间中添加注释或其他图形对象,并不会影响模型空间或其他布局。而如果需要在布局中编辑模型,则可使用如下办法在视口中访问模型空间:

(1)双击浮动视口内部。

(2)单击状态栏上的Model按钮。

(3)在命令行输入:mspace(或别名ms)

从视口中进行模型空间后,可以对模型空间的图形进行操作。在模型空间对图形作的任何修改都会反映到所有图纸空间的视口以及平铺的视口中。

如果需要从视口中返回图纸空间,则可相应使用如下方法:

(1)双击布局中浮动视口以外的部分。

(2)单击状态栏上的Paper按钮。

(3)在命令行输入:pspace(或别名ps)。



\subsection{打开或关闭浮动视口}

新视口的缺省设置为打开状态。对于暂不使用、或不希望打印的视口,用户可以将其关闭。控制视口开关状态的方法为:

(1)快捷菜单:选择视口后单击右键,选择“Display Viewport Objects(显示视口对象)”

(2)“Properties(特性)”窗口:“On(开)”选项

(3)命令行:-vports


\subsection{控制视口的比例锁定}

一般情况下,布局的打印比例设置为1:1,并且在视口中缩放图纸空间对象的同时,也将改变视口比例。如果将视口的比例锁定,则修改当前视口中的几何图形时将不会影响视口比例,但此时大多数查看命令将无效,如VPOINT、DVIEW、3DORBIT、PLAN和VIEW等。锁定视口比例的方法为:

(1)快捷菜单:选择视口后单击右键,选择“Display Locked(显示锁定)”

(2)“Properties(特性)”窗口:“Display Locked(显示锁定)”选项

(3)命令行:-vports

\subsection{消隐打印视口中的线条}

如果图形中包括三维面、网格、拉伸对象、表面或实体,打印时可以让AutoCAD删除选定视口中的隐藏线。视口对象的“Hide Plot(消隐出图)”打印特性只影响打印输出,而不影响屏幕显示。

(1)快捷菜单:选择视口后单击右键,选择“Hide Plot(消隐出图)”

(2)“Properties(特性)”窗口:“Hide Plot(消隐出图)”选项

(3)命令行:-vports


\subsection{相对图纸空间比例缩放视图}

在图纸空间布局中工作时,标准比例因子代表显示在视口中的模型的实际尺寸与布局尺寸的比率,通常该比例为1:1,即模型在模型空间和图纸空间具有相同的尺寸。如果需要精确地比例缩放所打印的视图,则可改变该比例,具体方法为:

(1)“Properties(特性)”窗口:“Standard Scale(标准比例)”选项


\subsection{在图纸空间比例缩放线型}

在缺省情况下,布局中的视口显示线型时,即使在布局和浮动视口中按不同比例显示的对象具有相同的线型缩放比例,线型的比例也不受视口缩放比例的影响。

如果设置系统变量PSLTSCALE的值为0,则视口的缩放比例改变后,其中所显示的线型也将随之发生变化。

\section{重定义视口边界}

对于一个已有的视口,用户可重新定义其边界。该命令的调用方式为:

(1)工具栏:“Viewports(视口)”$\to$重定义边界

(2)快捷菜单:选择视口对象后单击右键,选择“Viewport Clip(视口剪裁)”项

(3)命令行:vpclip

调用该命令后,系统首先提示用户选择一个已有的视口:

\begin{verbatim}
Select viewport to clip:
\end{verbatim}

然后用户可通过选择一个图纸空间中的对象或指定多边形的顶点来定义新的边界:

\begin{verbatim}
Select clipping object or [Polygonal] <Polygonal>:
\end{verbatim}


\chapter{图形设置}

用户可以在图纸空间中使用图形设置命令“mvsetup”来控制和设置视图。该命令的调用方式为:

(1)命令行:mvsetup(或mvs)

如果在模型空间调用该命令,则系统提示是否启用图纸空间:

\begin{verbatim}
Enable paper space? [No/Yes] <Y>:
\end{verbatim}

\begin{compactenum}
\item 如果选择“Yes”选项,或是直接在图纸空间调用该命令,则系统给出如下选项:

\begin{verbatim}
Enter an option [Align/Create/Scale viewports/Options/Title block/Undo]:
\end{verbatim}

各选项作用如下:


(1)“Align(对齐)”:用于在视口中平移或旋转视图,也可将两个视口中的视图对齐。选择该项后系统提示如下:

\begin{verbatim}
Enter an option [Angled/Horizontal/Vertical alignment/Rotate view/Undo]:
\end{verbatim}

\begin{compactitem}
\item “Angled(角度)”:在视口中沿指定的方向平移视图。
\item “Horizontal(水平)”:在视口中平移视图,直到它与另一个视口中的基点水平对齐为止。
\item “Vertical alignment(垂直对齐)”:在视口中平移视图,直到它与另一个视口中的基点垂直对齐为止。
\item “Rotate view(旋转视图)”:在视口中围绕基点旋转视图。
\item “Undo(放弃)”	:放弃已执行的操作。
\end{compactitem}

\item 如果选择“No”选项,系统进一步提示用户指定单位类型、比例因子、图纸宽度和高度:

\begin{verbatim}
Enter units type [Scientific/Decimal/Engineering/Architectural/Metric]: 
Enter the scale factor: 
Enter the paper width: 
Enter the paper height: 
\end{verbatim}

该命令将根据用户的选择更改图形的单位类型、图形界限等设置,并自动绘制一个矩形边框(Polyline对象)来显示图形界限。

说明:“mvsetup”命令可透明地使用。
\end{compactenum}



(2)“Create(创建)”:可删除对象或创建视口。系统提示如下:

\begin{verbatim}
Enter option [Delete objects/Create viewports/Undo] <Create>:
\end{verbatim}

(3)“Scale viewports(缩放视口)”:设置视口的缩放比例因子。选择该项后系统将提示用户选择对象,如果选择对象不只一个,则进一步提示如下:

\begin{verbatim}
Select objects:
Set zoom scale factors for viewports.  Interactively/<Uniform>:
\end{verbatim}

使用“Interactively(交互)”选项可分别设置每个视口的比例;使用“Uniform(统一)”则对所有视口设置统一比例因子。

(4)“Options(选项)”:该项用于设置要插入标题栏的图层、插入标题栏后是否重置图形界限、指定转换后的图纸单位以及指定标题栏是插入还是附着的外部参照等,系统提示如下:

\begin{verbatim}
Enter an option [Layer/LImits/Units/Xref] <exit>:
\end{verbatim}

(5)“Title block(标题栏)”:该项用于删除对象、设置原点和插入可用的标题栏。系统提示如下:

\begin{verbatim}
Enter title block option [Delete objects/Origin/Undo/Insert] <Insert>:
\end{verbatim}

(6)“Undo(放弃)”:放弃当前MVSETUP任务中已执行的操作。

说明:“mvsetup”命令可透明地使用。

\chapter{绘图比例}

最好使用1∶1比例绘图,输出比例可以随便调整。绘图比例和输出比例是两个概念,输出时使用“输出1单位=绘图500单位”就是按1/500比例输出,若“输出10单位=绘图1单位”就是放大10倍输出。

用1∶1比例画图好处很多。第一、容易发现错误,由于按实际尺寸画图,很容易发现尺寸设置不合理的地方。第二、标注尺寸非常方便,尺寸数字是多少,软件自己测量,万一画错了,一看尺寸数字就发现了(当然,软件也能够设置尺寸标注比例,但总得多费工夫)。第三、在各个图之间复制局部图形或者使用块时,由于都是1∶1比例,调整块尺寸方便。第四、由零件图拼成装配图或由装配图拆画零件图时非常方便。第五、用不着进行烦琐的比例缩小和放大计算,提高工作效率,防止出现换算过程中可能出现的差错。 




\chapter{打印比例}

在绘制图纸完毕后都要将图纸打印出来,只有当图纸打印出来(白图或硫酸纸晒蓝图)后,才可以认为绘图工作基本完毕(当然还有整理归档等)。

打印时,有一环不可回避,就是比例问题。

打印比例和绘图比例合理,那么最后完成的图纸就清晰美观漂亮。我们常常看到有的图纸密密麻麻,有的图纸空空荡荡,这些都是属于比例控制不当。那么就打印比例和绘图比例而言,我们在绘图时和绘图初期又应该注意什么呢? 

一般来说,我们最终打印的图纸有按照比例和不按照比例两种。按照比例的,一般是施工图;不按照比例的情况有很多,一般有方案文本、过程图(自己看的)、条件图等,最近几年的初步设计(扩初设计)文本也逐渐变成不按照比例打印成A3大小。 

先来看看按照比例打印的施工图。施工图基本上都是按比例出图的,但是在出图时会有不同的比例设置。比如说,平立剖一般都是1:100,楼梯间、卫生间是1:50,节点大样是1:20,装修图中的节点大样则可能会有1:10、1:5、1:2等的比例,总图的常用比例是1:500,有时候也会有1:1000、1:2000等。

接下来以一套包括平立剖面图、楼梯间、卫生间大样和节点大样的常规的建筑施工图来举例说明。 

首先,我们一开始绘制的一般都是平面图,这时应该以实际每1mm对应AutoCAD中的每个标准单位1来绘制,接下来的其他立剖面也同样。

这时有两个地方是应该要注意的,第一是所有的字体基本上都应设置为350高度左右,房间名可稍大些,图纸名称等专用字体高度另定。第二,就是在进行尺寸标注后,将该尺寸标注的“标注全局比例”设置成100。其他的设置,按照前文所述。 

接下来是1:50比例的楼梯大样和卫生间大样。在绘制这样的图时,一般先将平面图中的相关部分拷贝到原来图纸旁边,然后删除和剪接掉多余的部分。接下来要修改一些东西了。

\begin{compactitem}
\item 一是线性比例,除实线外,其他的如虚线、点划线等都改为原来一半的线性比例。
\item 二是字大小、标高大小、轴线号大小同样的Scale为原来的0.5比例。
\item 三是标注设置,将这一部分的所有标注中的“标注全局比例”修改为50。
\end{compactitem}

这样的话,按照1:50的比例打印出来时,各部分的大小尺寸都会比较合适。 

1:20的节点大样绘制方法一样,不再赘述。 

图纸绘制完毕后,要做的就是套图框。每个公司都有自己的图框,而且不同的图幅应该都有。

图框在制作时,大多都会按照1:1的比例:A0—1194*840;A1—840*597;A2—597*420;A3—420*297;A4—297*210。其中,A1和A2图幅的还经常用到竖图框。还有,我们需要用到加长图框时,应该是在图框的长边方向,按照图框长边1/4的模数增加。每个图框不管图幅是多少,按照一定的比例打印出来时,图签栏的大小都应该是一样的。我们把不同大小的图框按照我们要出图的比例Scale大,将图套在其中即可。 

关于施工图按照比例出图,还有两点要说。

第一点是并非所有的平立剖面图都要按照1:100的比例来,对于一些没什么细部的仓库、厂房时,也可以考虑用到1:150或1:200的比例。这样才能够使图纸饱满。 

第二点,就是不管是什么图,绘制时都切记以实际每1mm对应AutoCAD中的每个标准单位1来绘制(总图可以除外),不同比例的出图只是在字体大小、线性比例、标注的标注全局比例等地方不同。 

当然,以上情况是针对同一比例的图都放到同一个图框中出图的情况。对于不同比例的图要放到一个图框中时,稍微有一点不同。以一张1:100平面图中要放一张1:20的节点大样来举例如下。 

首先确定1:100为这张图中的主打印比例,1:100的平面图按照以上方法绘制完毕,1:20的节点大样,也按照上述方法绘制好,并修改好各个设置。当发现要将其并入1:100的图中出图时,把1:20的节点拷贝一份到1:100的图框内,Scale 5倍,然后需要修改两样东西。一是线性比例,二是标注中的两个参数。其中之一是标注全局比例改为1:100(即主打印比例),第二是标注线性比例,改为0.2。这样就OK了。当然,在绘制其他比例的图纸时,如果一开始就确定其要和其他不同比例的图纸放在一起,上面的有些顺序可以调整,可以减少一些步骤,较少耗时。 

对于不按照比例出的图,也同样有些东西要注意,同样也要注意比例问题。 

下面以一个方案来举例说明。 

我们在进行一个方案设计时,到了整理出图阶段,就要注意比例问题了。首先要确定一下最终出图的图幅,以前的方案文本都是A3的,现在也常用A2的了(即两个竖版A3拼)。在确定了最终出图比例后,要预估一下大概的出图比例,然后按照打印出来的字高为2.5mm左右(方案文本的比例一般在1:200以上,除非是做别墅、小住宅、小公建,字体大小和标注尺寸可以适当小些)凡推出在图纸中的字体大小和标注尺寸的合适数值。有必要时,可以在整理初期先将第一张画得差不多的图打出来看看,确认没问题后,再继续其他图纸的整理完善。图纸全部整理完毕,套好图框,交给打印公司就可以了。

在这里,稍微提一下图纸空间(Paper Space)。图纸空间是一样很有意思的东西,在图纸的比例设置、批量打印等方面都有很大的不同。


由于建筑专业绘制的图纸不光是自己出图用的,还要作为其他专业的条件图。在将图纸空间中的图纸提给其他专业后,结果是一个个不断的电话询问,这时发现,不光自己要掌握图纸空间,还要教会其他所有的专业掌握图纸空间。

不过在一些国外公司,除了在方案,在施工图中,大量运用到图纸空间。


\chapter{打印输出}


使用AutoCAD创建图形之后,通常要打印到图纸上,或者生成一份电子图纸。打印的图形可以包含图形的单一视图,或者更为复杂的视图排列。根据不同的需要,可以打印一个或多个视口,或设置选项以决定打印的内容和图像在图纸上的布置。

布局在图纸的可打印区域显示图形视图,模拟在纸面上绘图的情形。布局选项卡中显示实际打印的内容,还存储页面设置,包括打印设备、打印样式表、打印区域、旋转、打印偏移、图纸大小和缩放比例等。

所有的对象和图层都具有打印样式。使用打印样式能够改变图形中对象的打印效果。打印样式是一系列颜色、抖动、灰度、笔指定、淡显、线型、线宽、端点样式、连接样式和填充样式的替代设置。

用户可以在模型空间中或任一布局调用打印命令来打印图形,该命令的调用方式为:

(1)工具栏:“Standard(标准)”$\to$打印

(2)菜单:【File(文件)】$\to$【Plot…(打印)】

(3)快捷菜单:在模型或布局选项卡上单击右键弹出快捷菜单,并选择“Plot…(打印)”项。

(4)命令行:plot(或别名print)

调用该命令后,系统将弹出“Plot(打印)”对话框,该对话框的内容与“Page Setup(页面设置)”对话框类似。

(1)“Plotter Configuration”和“Plot style table”栏与“Page Setup(页面设置)”对话框中相同。

\begin{compactitem}
\item “Plot Stamp(打印标记)”:选择该项后,可在图形中指定的位置打印标记。用户可单击按钮,弹出“Plot Stamp(打印标记)”对话框,用于指定用于打印标记的信息(包括图名、日期和时间、打印比例等,也可自定义其他信息)和位置等。

\item “What to plot(打印内容)”:指定打印内容和打印份数。其中打印内容为以下三种情况之一:

\begin{compactenum}
\item “Current tab(当前选项卡)”:打印当前的“模型”或布局选项卡。
\item “Selected tabs(选定的选项卡)”:打印多个预先选定的选项卡。
\item “All layout tabs(所有布局选项卡)”:打印所有布局选项卡,无论选项卡是否选定。
\end{compactenum}

注意:如果选择了多个布局和副本,设置为“打印到文件”或“后台打印”的任何布局都只单份打印。

\item “Plot to file(打印到文件)”:选择该项后,系统将打印输出到文件而不是输出到打印机。用户需指定打印文件名和打印文件存储的路径。缺省的打印文件名为图形及选项卡名,用连字符分开;缺省的位置为图形文件所在的目录。
\end{compactitem}

(2)在“打印”对话框的“页面设置”下拉列表框中可以选择所要应用的页面设置名称,如果没有进行页面设置,选择“无”选项;

(3)在“打印机/绘图仪”选项组的“名称”下拉列表框中可以选择要使用的绘图仪;

(4)在“图纸尺寸”下拉列表框中可以选择合适的图纸幅面,并且在右上角可以预览图纸幅面的大小;

(5)在“打印区域”选项组的“打印范围”下拉列表框中提供了4种方法来确定打印范围,4种方法及其含义如下。

\begin{compactitem}
\item “图形界限”:表示打印布局时,将打印指定图纸尺寸的页边距内的所有内容,其原点从布局中的(0,0)点计算得出。
\item “显示”:表示打印选定的“模型”选项卡当前视口中的视图或布局中的当前图纸空间视图。
\item “窗口”:表示打印指定的图形的任何部分,这是直接在模型空间打印图形时最常用的方法。选择该选项后,命令行会提示用户在绘图区指定打印区域。
\item “范围”:用于打印图形的当前空间部分(该部分包含对象),当前空间内的所有几何图形都将被打印,该选项在图纸空间打印时出现,在模型空间打印时不出现。
\end{compactitem}

\begin{figure}[htbp]
\centering
\includegraphics{Print.png}
\caption{“打印”对话框}
\end{figure}

在“打印比例”选项组中,当选中“布满图纸”复选框后,其他选项显示为灰色表示不能更改。取消“布满图纸”复选框,用户将可以对比例进行设置。

单击“打印”对话框右下角的按钮,则展开“打印”对话框。在“打印样式表”下拉列表框中可以选择合适的打印样式表,在“图形方向”选项组中可以选择图形打印的方向和文字的位置。

单击“预览”按钮可以对打印图形效果进行预览,若对某些设置不满意可以返回修改。在预览中,按Enter键可以退出预览返回“打印”对话框,单击“确定”按钮即可进行打印。

除了使用“plot”命令进行打印之外,用户也可在“Page Setup(页面设置)”对话框中单击Plot按钮直接进行打印。

说明:“plot”命令具有相应的命令行形式“-plot”。

\chapter{打印样式}

打印样式(Plot style)是一种对象特性,用于修改打印图形的外观,包括对象的颜色、线型和线宽等,也可指定端点、连接和填充样式,以及抖动、灰度、笔指定和淡显等输出效果。

打印样式可分为“Color Dependent(颜色相关)”和“Named(命名)”两种模式。颜色相关打印样式以对象的颜色为基础,共有255种颜色相关打印样式。在颜色相关打印样式模式下,通过调整与对象颜色对应的打印样式可以控制所有具有同种颜色的对象的打印方式。

命名打印样式可以独立于对象的颜色使用。可以给对象指定任意一种打印样式,不管对象的颜色是什么。

打印样式表用于定义打印样式。根据打印样式的不同模式,打印样式表也分为颜色相关打印样式表和命名打印样式表。颜色相关打印样式表以“.ctb”为文件扩展名保存,而命名打印样式表以“.stb”为文件扩展名保存,均保存在AutoCAD系统主目录中的“plot styles”子文件夹中。

\section{创建打印样式表}

AutoCAD提供了两种向导,分别用于创建命令打印样式表和颜色相关打印样式表。

\subsection{创建命令打印样式表}

选择菜单【Tools(工具)】$\to$【Wizards(向导)】$\to$【Add Plot Style Table…(添加打印样式表)】,系统弹出“Add Plot Style Table(添加打印样式表)”对话框。

下面依次对各个步骤进行介绍。
1、“Begin(开始)”:选择如下创建方式之一:

\begin{compactitem}
\item “Start from scratch(创建新打印样式表)”:从头开始创建新的打印样式表。
\item “Use an existing plot style table(使用现有打印样式表)”:以现有的命名打印样式表为基础来创建新的打印样式表。
\item “Use My R14 Plotter Configuration(CFG)(使用R14打印机配置)”:使用acadr14.cfg文件中的笔指定信息创建新的打印样式表。
\item “Use a PCP or PC2 file(使用PCP或PC2文件)”:使用PCP或PC2文件中存储的笔指定信息创建新的打印样式表。
\end{compactitem}

2、“Table Type(表类型)”:选择创建命名打印样式表或者是创建颜色相关打印样式表。

3、“Browse File(浏览文件)”:如果要从已存在的文件、或CFG、PCP、PC2等文件中输入信息,需要在本步骤中进行定位。

4、“File name(文件名)”:指定新建的打印样式表名称。

5、“Finish(结束)”:在完成创建工作前,用户还可单击Plot Style Table Editor...按钮,用打印样式表编辑器(Plot Style Table Editor)对该文件进行编辑。

如果用户选择对话框中的“Use this plot style table for new and pre-AutoCAD xxxx drawings”项,则可按缺省规定附着打印样式到所有新图形和早期版本的图形中。

完成上述步骤后,系统将创建一个新的STB文件,并将其保存在AutoCAD系统主目录中的“plot styles”子文件夹中。



\subsection{创建颜色相关打印样式表}

选择菜单【Tools(工具)】$\to$【Wizards(向导)】$\to$【Add Color-Dependent Plot Style Table…(添加颜色相关打印样式表)】,系统弹出“Add Color-Dependent Plot Style Table(添加颜色相关打印样式表)”对话框。

下面依次对各个步骤进行介绍:

1、“Begin(开始)”:选择如下创建方式之一:

\begin{compactitem}
\item “Start from scratch(创建新打印样式表)”:从头开始创建新的打印样式表。
\item “Use a CFG file(使用CFG文件)”:使用acadr14.cfg文件中的笔指定信息创建新的打印样式表。
\item “Use a PCP or PC2 file(使用PCP或PC2文件)”:使用PCP或PC2文件中存储的笔指定信息创建新的打印样式表。
\end{compactitem}

2、“Browse File(浏览文件)”:如果要从CFG、PCP或PC2等文件中输入信息,需要在本步骤中进行定位。

3、“File name(文件名)”:指定新建的打印样式表名称。

4、“Finish(结束)”:在完成创建工作前,用户还可单击Plot Style Table Editor...按钮,用打印样式表编辑器(Plot Style Table Editor)对该文件进行编辑。

用户还可以选择“Use this plot style table for the current drawing”项将新建的打印样式表应用于当前图形。如选择“Use this plot style table for new and pre-AutoCAD xxxx drawings”项,则可按缺省规定附着打印样式到所有新图形和早期版本的图形中。

完成上述步骤后,系统将创建一个新的CTB文件,并将其保存在AutoCAD系统主目录中的“plot styles”子文件夹中。



\section{打印样式管理器}

打印样式管理器可以帮助用户创建、编辑和存储CTB和STB文件。启动打印样式管理器的方式为:

(1)菜单:【File(文件)】$\to$【Plot Style Manager…(打印样式管理器)】

(2)命令行:stylesmanager

(3)其他方式:操作系统中的控制面板$\to$“Autodesk Plot Style Manager”项

(4)“Options(选项)”$\to$“Plotting(打印)”选项卡$\to$Add or Edit Plot Style Tables

启动打印样式管理器,实际上是在操作系统的资源管理器中访问AutoCAD系统主文件夹中的“plot styles”子文件夹。


\section{编辑打印样式}

AutoCAD提供了打印样式表编辑器(Plot Style Table Editor),用以对打印样式表中的打印样式进行编辑。用户可使用如下方式来启动该编辑器:

(1)启动打印样式管理器,并打开其中的打印样式表文件(包括STB文件和CTB文件)。

(2)在“Plot(打印)”或“Page Setup(页面设置)”对话框中选择打印样式表并单击Edit按钮。

(3)在“Current Plot Style(当前打印样式)”或“Select Plot Style(选择打印样式)”对话框中选择打印样式表,并单击Editor按钮。

在该编辑器中,用户可以查看或设置打印样式。

根据打印样式表模式的不同,打印样式编辑器的功能也有所不同。比如可以在命名打印样式表中添加或删除打印样式,而在颜色相关打印样式表中包含的255个打印样式分别映射255种颜色,所以不能将新的样式添加到颜色相关打印样式表,也不能从颜色相关打印样式表中删除打印样式。


\section{应用打印样式}

每个AutoCAD的图形对象以及图层都具有打印样式特性,其打印样式的特性与所使用的打印样式的模式相关。

如果工作在颜色相关模式下,打印样式由对象或图层的颜色确定,所以不能修改对象或图层的打印样式。

如果工作在命名打印样式模式下,则可以随时修改对象或图层的打印样式。可用的打印样式有如下几种:

\begin{compactenum}
\item “Normal(普通)”:使用对象的缺省特性。
\item “ByLayer(随层)”:使用对象所在图层的特性。
\item “ByBlock(随块)”:使用对象所在块的特性。
\item 命名打印样式:使用在打印样式表中定义打印样式时指定的特性。
\end{compactenum}

创建对象和图层时,AutoCAD 为其指定当前的打印样式。如果插入块,则块中的对象使用它们自己的打印样式。

在“Option(选项)”对话框中的“Plotting(打印)”选项卡中,用户可以选择新建图形所使用的打印样式模式。

下面左图和右图分别显示了选择颜色相关模式和命名模式两种情况。其中,在命名模式下,还可进一步设置“0”层和新建对象的缺省打印样式。

\begin{figure}[htbp]
\centering
\includegraphics{default_plot_style.png}
\caption{选择颜色相关模式与选择命名模式}
\end{figure}

\chapter{退出AutoCAD}

用户退出AutoCAD的方法有如下几种:

1、打开File菜单,选取Exit项。若用户的当前文件没有保存,AutoCAD将显示是否保存文件的提示。

\begin{compactitem}
\item 用户若选取“Yes”按钮或直接回车,AutoCAD将当前图形文件存盘之后退出AutoCAD。
\item 用户若选取“No”按钮,AutoCAD将不保存当前图形直接退出AutoCAD。
\item 用户若选取“Cancel”按钮,则取消退出AutoCAD的操作。
\end{compactitem}

2、单击右上角的关闭按钮,同样可以退出AutoCAD。

若用户没有保存当前的图形文件,AutoCAD仍会给出前面是否保存文件的提示。

3、利用命令行输入QUIT命令,同样也可以退出AutoCAD。

\begin{verbatim}
命令:QUIT(或EXIT)
\end{verbatim}

若用户没有保存当前的图形文件,AutoCAD仍会给出前面是否保存文件的提示。

4、利用命令行输入END命令,此时若用户没有保存当前的图形文件,AutoCAD将不给任何提示,直接退出AutoCAD。建议用户不要选取这种方法。(END命令在AutoCAD 2008以及后续版本中已经废除,建议使用QUIT或EXIT命令来保存文件退出)

5、使用快捷键【Alt+F4】可以退出AutoCAD。

